% Bloque de Código LaTeX para las Figuras

% --- Figura 1: Dinámica Laminar ---
\begin{figure}[h!]
    \centering
    \includegraphics[width=0.9\textwidth]{images/laminar_dynamics.png}
    \caption{Dinámica Laminar de una Columna Cortical simulada. La figura ilustra la integración jerárquica de corrientes sinápticas, donde el estímulo inicial en L4 se propaga a las capas supragranulares (L2/3) y, finalmente, a las infragranulares (L5). Se observa claramente el tiempo de recuperación post-disparo (periodo refractario) en las neuronas de L5, evidenciando la consistencia del modelo con la dinámica neuronal canónica.}
    \label{fig:laminar_dynamics}
\end{figure}

% --- Figura 2: Mapeo CBN ---
\begin{figure}[h!]
    \centering
    \includegraphics[width=0.9\textwidth]{images/cbn_mapping.png}
    \caption{Mapeo de la Dinámica Continua a la Red Booleana Causal (CBN). Se muestra la correspondencia entre los spikes del modelo LIF (arriba) y los estados discretos '1' de la CBN (abajo). El operador de proyección $\Phi_{\Delta t}$ demuestra preservar el \textit{spike timing} con alta fidelidad, a pesar de la discretización. Este resultado soporta la hipótesis de que el \textit{temporal coding} es un mecanismo eficiente que se mantiene incluso en esta abstracción de meso-escala, como lo describe Trappenberg.}
    \label{fig:cbn_mapping}
\end{figure}

% --- Figura 3: Análisis Espectral ---
\begin{figure}[h!]
    \centering
    \includegraphics[width=0.9\textwidth]{images/spectral_analysis.png}
    \caption{Análisis Espectral de la Actividad de la Red. El análisis de Fourier de la tasa de disparo promedio de la capa de salida (L5) revela una propiedad emergente de la red acoplada: una oscilación dominante en la banda Gamma. El pico de potencia se localiza precisamente en 42 Hz, un hallazgo consistente con la actividad cerebral registrada durante tareas de procesamiento cognitivo.}
    \label{fig:spectral_analysis}
\end{figure}
