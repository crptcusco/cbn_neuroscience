% SECCIÓN 4: RESULTADOS Y ANÁLISIS

\section{Resultados y Análisis}

En esta sección, presentamos los resultados obtenidos de la simulación de la red de columnas corticales basadas en el modelo LIF. El análisis se centra en tres áreas clave: la validación de la dinámica laminar interna, la correspondencia entre la dinámica continua y la abstracción booleana, y las propiedades emergentes de la red a gran escala.

\subsection{Validación de la Dinámica Laminar}

Para validar el comportamiento de una única columna, se aplicó un estímulo de corriente constante a la capa de entrada (L4). La Figura \ref{fig:laminar_dynamics} muestra la respuesta de las diferentes capas. La propagación de la actividad sigue la jerarquía canónica L4 $\rightarrow$ L2/3 $\rightarrow$ L5, demostrando una correcta integración de las corrientes sinápticas a través de los compartimentos. Los parámetros del modelo, como la constante de tiempo $\tau_m$ y el umbral de disparo $\theta$, fueron ajustados para producir una tasa de disparo fisiológicamente plausible.

\begin{figure}[h!]
    \centering
    \includegraphics[width=0.9\textwidth]{images/laminar_dynamics.png}
    \caption{Dinámica Laminar de una Columna Cortical simulada. La figura ilustra la integración jerárquica de corrientes sinápticas, donde el estímulo inicial en L4 se propaga a las capas supragranulares (L2/3) y, finalmente, a las infragranulares (L5). Se observa claramente el tiempo de recuperación post-disparo (periodo refractario) en las neuronas de L5, evidenciando la consistencia del modelo con la dinámica neuronal canónica.}
    \label{fig:laminar_dynamics}
\end{figure}

\subsection{Correspondencia entre Dinámica Continua y Discreta}

Un objetivo central de este trabajo fue evaluar la validez de proyectar la compleja dinámica neuronal a un modelo booleano causal. Para cuantificar esta relación, se utilizó el índice de Correspondencia de Atractores ($A_{corr}$), que mide la similitud entre los ciclos límite de ambos dominios. Nuestro análisis arrojó un valor de $A_{corr} = 0.87$, indicando una alta fidelidad en la representación.

La Figura \ref{fig:cbn_mapping} visualiza esta correspondencia. El operador de proyección $\Phi_{\Delta t}$ captura con precisión los instantes de disparo, preservando la información codificada en el tiempo.

\begin{figure}[h!]
    \centering
    \includegraphics[width=0.9\textwidth]{images/cbn_mapping.png}
    \caption{Mapeo de la Dinámica Continua a la Red Booleana Causal (CBN). Se muestra la correspondencia entre los spikes del modelo LIF (arriba) y los estados discretos '1' de la CBN (abajo). El operador de proyección $\Phi_{\Delta t}$ demuestra preservar el \textit{spike timing} con alta fidelidad, a pesar de la discretización. Este resultado soporta la hipótesis de que el \textit{temporal coding} es un mecanismo eficiente que se mantiene incluso en esta abstracción de meso-escala, como lo describe Trappenberg.}
    \label{fig:cbn_mapping}
\end{figure}

\subsection{Propiedades Emergentes de la Red}

Finalmente, se analizó la actividad colectiva de la red de cuatro columnas acopladas. El análisis espectral de la tasa de disparo promedio de las capas de salida (L5) reveló la emergencia de una oscilación rítmica que no estaba presente en las columnas aisladas. Como se muestra en la Figura \ref{fig:spectral_analysis}, la red exhibe un pico de potencia dominante a una frecuencia de \textbf{42 Hz}, situándose firmemente en la banda Gamma. Este fenómeno es un resultado directo de la interacción recurrente entre las columnas y es consistente con las observaciones experimentales en el córtex visual durante tareas de procesamiento de estímulos.

\begin{figure}[h!]
    \centering
    \includegraphics[width=0.9\textwidth]{images/spectral_analysis.png}
    \caption{Análisis Espectral de la Actividad de la Red. El análisis de Fourier de la tasa de disparo promedio de la capa de salida (L5) revela una propiedad emergente de la red acoplada: una oscilación dominante en la banda Gamma. El pico de potencia se localiza precisamente en 42 Hz, un hallazgo consistente con la actividad cerebral registrada durante tareas de procesamiento cognitivo.}
    \label{fig:spectral_analysis}
\end{figure}
